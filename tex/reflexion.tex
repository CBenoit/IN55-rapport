\section{Réflexions et réfractions sur le diamant}

\subsection{Réflexions}

L'objectif de cette partie était de trouver  différentes méthodes
de calcul de réflexion, afin d'avoir le meilleur réalisme visuel possible. \\

Dans une première approche, certes simpliste, mais rapide à mettre en œuvre,
nous avons utilisé une simple texture cubique. Cette méthode est extrêmement peu
coûteuse en performance, mais permet tout de même un rendu de qualité raisonnable
pour des objets simples. \\
Malheureusement, les diamants comportant des réflexions internes, cette méthode ne permet pas
de rendu convaincant, un rendu cubique autour de chaque face pouvant apporter des artefacts
visuels sur des faces adjacentes et proches.

Nous avons donc creusé dans une autre direction, en générerant, pour chaque triangle de la scène,
un rendu de son monde «~miroir~», dans un buffer commun à toute la scène.

Le but final est d'obtenir quelque chose ressemblant à ça~:
\includegraphics[width=0.45\textwidth]{Reflexion/6}
avec, en rouge, la zone où le rendu de la réflexion sera effectué.

Nous avons dans cet exemple un carré avec un miroir triangulaire tourné à 45° par rapport au mur.

\includegraphics[width=0.45\textwidth]{Reflexion/1}

Pour chaque triangle se comportant comme un miroir on calcul la réflexion de la manière suivante~:

\begin{itemize}
    \item On récupère sa matrice Tangente – Bitangente – Normale (TBN), qui nous permet
        d'effectuer la réflexion de la scène par rapport au triangle.

        \includegraphics[width=0.45\textwidth]{Reflexion/3}

    \item On supprime ce qui se trouve de l'autre côté du mur ; en effet, une
        fois la réflexion effectuée, c'est uniquement la partie à l'avant qui
        représente le reflet.

        \includegraphics[width=0.45\textwidth]{Reflexion/4}
        \includegraphics[width=0.45\textwidth]{Reflexion/5}
    \item Et pour finir, on ne garde que les pixels se trouvant à l'intérieur
        du triangle que l'on est en train de rendre.

        \includegraphics[width=0.45\textwidth]{Reflexion/6}
\end{itemize}

\subsection{Réfractions}

Pour calculer les réfractions, il y a une étape supplémentaires, car la lumière est dispersée, mais aussi décomposée,
en fonction de l'angle d'incidence et du matériau.

On a choisi ici de traiter l'image comme le ferait une vraie caméra~:
en trois canaux de couleur.
On effectue chaque rendu trois fois, ou on rendra de manière séparée, le bleu,
le vert, et le rouge. Chaque réfraction dépendra donc de la matrice TBN de la
surface, de son indice de réfraction, qui est le rapport entre l'angle d'incidence
et l'angle réfracté, et aussi de la longueur d'onde de la couleur
en train d'être rendue.

\includegraphics[width=0.65\textwidth]{Refraction/refraction}

Les étapes suivantes sont les mêmes que pour la réflexion.

Ceci étant l'étape triviale de notre algorithme, car il faut maintenant faire la
récurrence afin de voir les réflexions/réfractions sur plusieurs faces (la profondeur
de réflexion étant paramétrable).

Ce n'est pas une étape à prendre à la légère, il faut effectuer un rendu des
réflexions/réfractions pour chaque niveau de récursivité, et ainsi combiner les
textures lors d'un double reflet (Ex: un miroir étant reflété par un autre) en
prenant l'angle d'incidence en compte via l'utilisation du Fresnel.

Cette méthode est très coûteuse en termes de performance,
mais notre but étant le réalisme, cela est satisfaisant.

